\documentclass{emulateapj}
\bibliographystyle{apj}

%define general packages
\usepackage{epsfig}
\usepackage{rotating}
\usepackage{amsmath}
\usepackage{natbib}
\usepackage{footnote}
\usepackage{courier}

%internal short cuts
\def \HgA {H$\gamma_A$}
\def \Hbp {H$\beta ^\prime$}
\def \lowenvthresh {0.5}
\def \highenvthresh {3.0}
\def \apradius{2.5}
\def \apheight{35}


\begin{document}

\title{PRIMUS: Galaxy Environment on the Quiescent Fraction Evolution at $z < 0.8$}
\author{
ChangHoon~Hahn\altaffilmark{1}, 
Michael R.~Blanton\altaffilmark{1}, 
John~Moustakas\altaffilmark{2},
Alison L.~Coil\altaffilmark{3}, 
Richard J.~Cool\altaffilmark{4}, 
Daniel J.~Eisenstein\altaffilmark{5},
Ramin A.~Skibba\altaffilmark{3}
Kenneth C.~Wong\altaffilmark{6}, 
Guangtun~Zhu\altaffilmark{7}
}
\altaffiltext{1}{Center for Cosmology and Particle Physics, Department of Physics, New York University, 4 Washington Place, New York, NY 10003; chh327@nyu.edu}
\altaffiltext{2}{Department of Physics and Astronomy, Siena College, 515 Loudon Road, Loudonville, NY 12211}
\altaffiltext{3}{Center for Astrophysics and Space Sciences, Department of Physics, University of California, 9500 Gilman Dr., La Jolla, CA 92093}
\altaffiltext{4}{MMT Observatory, University of Arizona, 1540 E Second Street, Tucson AZ 85721}
\altaffiltext{5}{Harvard-Smithsonian Center for Astrophysics, 60 Garden Street, Cambridge, MA 02138}
\altaffiltext{6}{Steward Observatory, University of Arizona, 933 North Cherry Avenue, Tucson, AZ 85721} 
\altaffiltext{7}{Hubble Fellow; Department of Physics and Astronomy, The Johns Hopkins University, 3400 North Charles Street, Baltimore, MD 21218} 
%%%%%%%%%%%%%%%%%%%%%%%%%%%%%%%%%%%%%%%%%%%%%%%%%%%%%%%%%%%%%%%%%%%%%%%%%%%%%%%%%%%%
% ABSTRACT
%%%%%%%%%%%%%%%%%%%%%%%%%%%%%%%%%%%%%%%%%%%%%%%%%%%%%%%%%%%%%%%%%%%%%%%%%%%%%%%%%%%%
\begin{abstract}
We investigate the effects of galaxy environment on the evolution of
the quiescent fraction ($f_{\mathrm{Q}}$) from $z =0.8 $ to $ 0.0$ using
spectroscopic redshifts and multi-wavelength imaging data from the
PRIsm MUlti-object Survey (PRIMUS) and the Sloan Digitial Sky Survey
(SDSS). Our stellar mass limited galaxy sample consists of $\sim
14,000$ PRIMUS galaxies within $z = 0.2-0.8$ and $\sim 64,000$ SDSS
galaxies within $z = 0.05-0.12$. We classify the galaxies as quiescent
or star-forming based on an evolving specific star formation cut, and
as low or high density environments based on fixed cylindrical
aperture environment measurements on a volume-limited environment
defining population. For quiescent and
star-forming galaxies in low or high density environments, we examine
the evolution of their stellar mass function (SMF). Then using the
SMFs we compute $f_{\mathrm{Q}}(\mathcal{M}_{*})$ and quantify its
evolution within our redshift range. We find that the quiescent
fraction is higher at higher masses and in denser environments. The
quiescent fraction rises with cosmic time for all masses and
environments. At a fiducial mass of $10^{10.5}M_\odot$, from $z\sim
0.7$ to $0.1$, the quiescent fraction rises by $15\%$ at the
lowest environments and by $25\%$ at the highest environments we measure.
These results suggest that for a minority of galaxies their cessation
of star formation is due to external influences on
them. However, in the recent Universe a substantial fraction of the
galaxies that cease forming stars do so due to internal processes.
\end{abstract}
\section*{}
\end{document}