\documentclass{emulateapj}
\bibliographystyle{apj}

%define general packages
\usepackage{epsfig}
\usepackage{rotating}
\usepackage{amsmath}
\usepackage{natbib}
\usepackage{footnote}
\usepackage{courier}

%internal short cuts
\def \HgA {H$\gamma_A$}
\def \gon {Gonz\'{a}lez}
\def \Hbp {H$\beta ^\prime$}

\begin{document}

\title{PRIMUS: Quiescent Fraction as a Function of Environment and Redshift}
\author{ChangHoon Hahn, Michael Blanton, Alison Coil, Richard Cool, Daniel Eisenstein, John Moustakas, Ken Wong, Guangtun Zhu}

\begin{abstract}
We present the evolution of the quiescent fraction ($f_{\rm{Q}}$) for galaxies in 
different density environments from $z=0.0$ to $0.8$ using spectroscopic 
redshift and multi-wavelength imaging data from PRism MUlti-object 
Survey (PRIMUS) and the Sloan Digitial Sky Survey (SDSS). We construct 
a stellar mass limited galaxy population of 
$\sim 40000$ galaxies from PRIMUS within the redshift range $0.2-0.8$ and $\sim 100000$
galaxies from SDSS within redshift range $0.0375-0.145$. Using an evolving cut based on
specific star formation, we classify the galaxies as quiescent or star-forming and
measure its environment using a fixed cylindrical aperture method 
on a volume-limited {\em Environment Defining Population} we construct 
from PRIMUS and SDSS. Based on its environment we classify our target
population into dense or sparse environments.   

With the target galaxy population divided into quiescent and star-forming galaxies 
in dense and sparse environments, we calculate the stellar mass functions 
(SMFs) for each of these subsamples. % should I include any conclusions on SMF in the abstract? 
With these SMFs, we compute the $f_{\rm{Q}}$s to find that at 
$\mathcal{M}_{*} \sim 10.5$, $f_{\rm{Q}}$ increases by $\sim 0.15$ from 
$z=0.8$ to $0.0$ for both high and low environments. 
In addition, throughout the redshift range the difference between the
$f_{\rm{Q}}$ at high and low environment remains constant at $< 0.1$. 
These results suggest that while $f_{\rm{Q}}$ increases for galaxies in 
both dense and sparse environments over redshift, the evolution of $f_{\rm{Q}}$ 
is independent of environment. 
\end{abstract}

%Section heading
\section{Introduction}
Galaxies, in their detailed properties, carry the imprints of their
surroundings, with a strong dependence of the quiescent fraction of
galaxies on their local environment (e.g. \citealt{hubble36a,
oemler74a, dressler80a, hermit96a, guzzo97a}; for a recent review see
\citealt{blanton09a}).  The strength of this dependence is itself a
strongly decreasing function of galaxy stellar mass; at the extreme,
the lowest masses ($<10^{9}$ $M_\odot$) galaxies are quenched only in
dense regions, and never in isolation (\citealt{geha12a}).  These
effects vary with redshift at least in the densest clusters, as
observed in the changing fraction of late-type spirals relative to the
field found in studies of the morphology-density relation
(\citealt{dressler84a, desai07a}).  Clearly understanding the
properties of galaxies in the present-day universe requires a careful
investigation of the role of environment, and how that role changes
over time.

Nevertheless, the evolution of the role of environment is a relatively
subtle effect and difficult to study.  Although history of galaxies
prior to $z\sim 1$ appears to have been one of rapid assembly, since
that time the galaxy population has continued to evolve, but less
dramatically. Although there are detectable changes in the population,
the major classes of galaxies existed at $z\sim 1$, in roughly the
same relative numbers as today (\citealt{bundy06a, borch06a,
taylor09a, moustakas13a}. Furthermore, at those redshifts we can also
detect the dependence of galaxy properties on environment, with lower
star-formation rate early-type galaxies populating the denser regions
(\citealt{cooper08a, patel09a, kovac10a}).

The most dramatic change in galaxy properties during the past eight
billion years has been a remarkable decline in the star-formation rate
of galaxies in the Universe (\citealt{hopkins06a}).  This decline
appears dominated by decreases in the rates of star-formation of
individual galaxies (\cite{Noeske:2007aa}). There is evidence that a
large fraction of the decline associated with strongly
infrared-emitting starbursts (\citealt{bell05a, magnelli09a}).  The
decline does not appear to be due to the quenching of a large
fractions of the star-forming population, as reflected in observations
of the stellar mass function of quiescent and star-forming galaxies
(\cite{Blanton:2006aa},\citealt{bundy06a, borch06a, moustakas13a}).  These
findings leave little room for the participation of
environmentally-driven quenching in the global census of
star-formation.  As \citet{cooper08a} and others have pointed out,
because the environmental dependence of total star-formation rates at
fixed redshift is relatively small, environmentally effects are
unlikely to cause the overall star-formation rate decline.

Thus, the impact of environment on galaxy formation has to be
interpreted on top of the background of this overall decline affecting
galaxies in all environments.  The most straightforward investigation
of would directly determine the star-forming properties of galaxies as
a function of environment, stellar mass and redshift in a single,
consistently analyzed data set. This analysis can reveal how galaxies
are quenched in the universe over time, quantitatively establish the
contribution of environmental effects to the overall trends, and
reveal whether those trends happen equally in all environments.
However, such an analysis has not been done previously due to the lack
of sufficiently large samples. In this paper, we apply this approach
using the Prism Multi-object Survey (PRIMUS; \cite{Coil:2011aa}, 
\cite{Cool:2013aa}), the largest available redshift survey covering the epochs
between $0<z<1$.

\section{Sample Selection} \label{sec:sample}
In this paper we are interested in measuring the evolution of the quiescent 
fraction over a wide range of redshifts and in different galaxy environments. 
In order construct a sample with redshift depth and robust enough redshift values 
to measure galaxy environment, we use galaxies at intermediate redshifts from PRIMUS. 
Additionally we supplement our sample with galaxies at low redshift ($z \sim 0.1$) from SDSS.

We begin with a brief summary of the PRIMUS data in Section \ref{sec:primus} followed by a summary of the SDSS data in Section \ref{sec:sdss}.
Then in Section \ref{sec:target} we use this data to define the stellar mass complete target galaxy population.
Afterwards, in Section~\ref{sec:sfq}, we classify these target galaxies as quiescent or active star-forming galaxies. 
For each galaxy we obtain the environment using a volume-limited {\em Environment Defining Population} in Section \ref{sec:environment}. 
Finally in Section \ref{sec:edgeeffect}, we correct the galaxy sample and its environment measurements for the edge effects of the surveys. 

\subsection{PRIMUS} \label{sec:primus}
For galaxies at intermediate redshifts we use multiwavelength imaging and spectroscopic redshifts data of PRIMUS, which is a faint galaxy survey with precise spectroscopic redshifts ($\sigma_z/(1+z) \approx 0.5 \%$) for $\sim 120,000$ galaxies within redshifts $z \approx 0-1.2$.
The survey was conducted using a IMACS spectrograph on a Magellan I Baade $6.5$ m telescope with a slitmask and low dispersion prism.
For further details on the PRIMUS observation methods, including survey design, targeting, and data summary, we refer readers to the survey papers: \cite{Coil:2011aa} and \cite{Cool:2013aa}.

As done in \cite{Moustakas:2013aa}, we only use fields targeted by PRIMUS with $GALEX$ and {\em Spitzer}/IRAC imaging.
This restricts us to five fields.
Four of these fields are a part of the {\em Spitzer} Wide-area Infrared Extragalactic Survey (SWIRE\footnote{http://swire.ipac.caltech.edu/swire/swire.html} ): 
the European Large Area ISO Survey - South $1$ field (ELAIS-S1\footnote{http://dipastro.pd.astro.it/esis}), the Chanddra Deep Field South SWIRE field (CDFS), 
and the XMM Large Scale Structure Survey field (XMM-LSS).
The XMM-LSS consists of two separate but spatially adjacent fields: the Subaru/XMM-Newton DEEP Survey fied (XMM-SXDSS\footnote{http://www.naoj.org/cience/SubaruProject/SDS})
and the Canadian-France-Hawaii Telescope Legacy Survey field (XMM-CFHTLS\footnote{http://www.cfht.hawaii.edu/Science/CFHLS}).
In addition to the SWIRE fields we also include the COSMOS\footnote{http://cosmos.astro.caltech.edu} field for a total of five fields. 

In all of the PRIMUS target fields we have near-UV (NUV) and far-UV (FUV) measurements from the {\em GALEX} Deep Imaging Survey (DIS; \cite{Martin:2005aa}; \cite{Morrissey:2005aa}). 
To minimize contamination from neighboring sources, we use a Bayesian photometric code EM$_{\rm{PHOT}}$ (based on expectation maximization algorithm of \cite{Guillaume:2006aa}). 
Furthermore, we use ground-based optical and {\em Spitzer}/IRAC mid-infrared photometric catalogs in each of the fields to obtain integrated fluxes in all 
photometric bands.
To summarize, the general strategy employed is to use a circular aperture photometry to constrain the shape of the SED and then fixing the overall normalization to 
a estimate of the total magnitude in the detection band. 
\cite{Moustakas:2013aa} provides a detailed description of the calculation for each of our target fields.  

From the spectroscopic redshift and photometry described above, we use \texttt{iSEDfit} to determine stellar masses, star formation rates (SFRs) and other physical properties
in a simplified Bayesian framework.
\texttt{iSEDfit}, which we will only briefly mention in this paper is discussed in detail in Appendix A. of \cite{Moustakas:2013aa}.
The code uses the redshift and the observed photometry of the galaxies to determine the statistical likelihood of a large ensemble of generated model SEDs. 
These generated model SEDs depend on population synthesis models and prior parameters.
In order to derive our fiducial stellar masses and star formation rates, we use the Flexible Stellar Population Synthesis (FSPS) models (\cite{Conroy:2010aa}) based on the \cite{Chabrier:2003aa} IMF. 
Other prior parameters are listed in Section 4.1 of \cite{Moustakas:2013aa}. 
The photometric bands we use for the fitting in our PRIMUS data are the {\em GALEX} FUV and NUV, the two shortest IRAC bands at $3.6$ and $4.5 \mu \rm{m}$, and the five optical bands
(in the COSMOS field, we fit seven optical bands and near-infrared bands, see Section 2.3 of \cite{Moustakas:2013aa}).
For a more detailed description of the data used in this paper, we refer readers to \cite{Moustakas:2013aa} Section 2 and Section 4.1.
\begin{table} %Subject to significant change based on the classification of environment
  \caption{Galaxy Subsamples}
  \label{tab:subsample}
  \begin{center}
    \leavevmode
    \begin{tabular}{lllll} \hline \hline              
  Survey    &Redshift ($z$) &Density        &Quiescent  &Star-forming  \\ \hline 
  SDSS      &$0.0375-0.145$ &High           &5470                       &4501                           \\
            &               &Mid            &3614                       &4438                           \\
            &               &Low            &5419                       &8927                           \\
            &               &               &                       &                           \\ \hline
  PRIMUS    &$0.2-0.4$      &High           &322                    &583                           \\
            &               &Mid            &177                    &403                          \\
            &               &Low            &768                    &2516                           \\
            &               &               &                       &                           \\ \hline
  PRIMUS    &$0.4-0.6$      &High           &350                       &675                           \\
            &               &Mid            &195                       &405                           \\
            &               &Low            &871                       &2385                           \\
            &               &               &                       &                           \\ \hline
  PRIMUS    &$0.6-0.8$      &High           &347                       &430                           \\
            &               &Mid            &186                       &327                           \\
            &               &Low            &833                       &1847                           \\
            &               &               &                       &                           \\ \hline
  PRIMUS    &$0.8-1.0$      &High           &136                       &232                           \\
            &               &Mid            &94                       &163                           \\
            &               &Low            &373                       &810                           \\
            &               &               &                       &                           \\ \hline
  \multicolumn{5}{l}{}                                             \\       
    \end{tabular} 
  \end{center}
\end{table}

\begin{figure*}
    \begin{center}
        \leavevmode
        \label{fig:targetEDP}
        \epsscale{1.0}
	\plotone{fig_EDP_absmag_redshift.png}
        \caption{Absolute magnitude $M_{r}$ versus redshift for the target galaxy population (black) with the 
Environment Defining Population (red) plotted on top. Both samples are divided into redshift bins:$0.0375-0.145$, 
$0.2-0.4$, $0.4-0.6$, and $0.6-0.8$. Stellar mass completeness limits are imposed on the target galaxy population (Section \ref{sec:target}) 
while the $M_{r}$ limits are imposed on the EDP (Section \ref{sec:environment}).}
    \end{center}
\end{figure*}

\subsection{SDSS-GALEX} \label{sec:sdss}
For galaxies at low redshifts we use the SDSS Data Release 7 (DR7; \cite{Abazajian:2009aa}). 
From the SDSS DR7 data, which provides high fidelity $ugriz$ photometry and spectroscopic redshift, we specifically use galaxies from the New York University 
Value-Added Galaxy Catalog that satisfy the main sample criterion and have galaxy extinction corrected Petrosian magnitudes $14.5 < r < 17.6$ 
and spectroscopic redshits within $0.01<z<0.2$ (\cite{Blanton:2005aa}). 
Within this sample, we select galaxies with medium depth observations from GALEX. 
This is done by first retrieving the positions of all {\em GALEX} tiles from GALEX Release 6 with total exposure time greater than $1$ ks and then constructing 
a joint angular selection function of the SDSS-{\em GALEX} Sample. 
This results in a final sample of $167,727$ SDSS galaxies with {\em GALEX} imaging. 

For this final sample, we use the MAST/CasJobs\footnote{http://galex.stsci.edu/casjobs} interface and a $4''$ diamater search radius to obtain the NUV and FUV photometry. 
For the optical photometry we use $ugriz$ bands (from the SDSS \texttt{model} magnitudes) scaled to the $r$-band \texttt{cmodel} magnitude. 
We supplement the above UV and optical photometry with integrated $JHK_s$ magnitudes from the 2MASS Extended Source Catalog (XSC; \cite{Jarrett:2000aa}) and with photometry at $3.4$ 
and $4.6 \mu \rm{m}$ from the WISE All-Sky Data Release\footnote{http://wise2.ipac.caltech.edu/docs/release/allsky}. 
Further details on the photometry of the SDSS data used in this paper is provided in \cite{Moustakas:2013aa} Section 2.4. 

To obtain the stellar masses and SFRs for the SDSS data we use \texttt{iSEDfit}, as we did for the PRIMUS data in Section \ref{sec:primus}.  
However, we use twelve photometric bands: {\em GALEX} FUV and NUV, SDSS $ugriz$, 2MASS $JHK_{s}$, and WISE $3.4$ and $4.6 \mu \rm{m}$.

\subsection{Target Galaxy Sample} \label{sec:target} 
In this section we will define the target galaxy sample 
used to compute the SMFs and QFs. The population 
at $z \sim 0.5$ is derived from PRIMUS data and the 
population at $z \sim 0.1$ is derived from SDSS-{\em GALEX} 
data, both described above. For the intermediate redshift 
population, we begin with the selection criteria imposed 
in \cite{Moustakas:2013aa} for their parent sample. We 
take the statistically complete {\em primary} sample from 
the PRIMUS data (\cite{Coil:2011aa}) and impose magnitude 
limits on optical selection bands as specified in 
\cite{Moustakas:2013aa} Table 1. These limits are in different 
optical selection bands and have distinct values for the 
five PRIMUS target fields. We then exclude stars and 
broad-line AGN to only select objects spectroscopically 
classified as galaxies, with high-quality spectroscopic redshifts 
($Q \geq 3$) within the redshift range $0.2 - 0.8$. For 
each of these objects we assign statistical weights (described 
in \cite{Coil:2011aa} and \cite{Cool:2013aa}) in order to correct 
for targeting incompleteness and redshift failures. The 
statistical weight, $w_i$, for each galaxy is given by
\begin{equation}
w_{i} = (f_{\rm{target}} \times f_{\rm{collision}} \times f_{\rm{success}})^{-1},
\end{equation}
Equation (1) in \cite{Moustakas:2013aa}. Then to account 
for stellar mass incompleteness, we impose derived stellar 
mass limits to the sample. 

Stellar mass completeness limits for a magnitude-limited 
survey such as PRIMUS is a function of redshift, apparent 
magnitude limit of the survey, and the typical stellar 
mass-to-light ratio of galaxies near the flux limit. As done 
in \cite{Moustakas:2013aa}, we follow \cite{Pozzetti:2010aa} 
to empirically determine the stellar mass completeness limits.
Briefly, for each of the target galaxies we compute 
$\mathcal{M}_{\rm{lim}}$ using $\rm{log} \; \mathcal{M}_{\rm{lim}} = \rm{log} \; \mathcal{M} + 0.4 ( m - m_{\rm{lim}})$.
$\mathcal{M}$ is the stellar mass of the galaxy in $\mathcal{M_{\odot}}$, 
$\mathcal{M}_{\rm{lim}}$ is the stellar mass of each galaxy 
if its magnitude was equal to the survey magnitude limit, 
$m$ is observed apparent magnitude in the selection band, 
and $m_{\rm{lim}}$ is the magnitude limit for our five 
fields mentioned above. We construct a cumulative 
distribution of $\mathcal{M}_{\rm{lim}}$ for the $15\%$ 
faintest galaxies in $\Delta z=0.04$ bins. In each of these 
redshift bins, we calculate the minimum stellar mass that 
includes $95 \%$ of the galaxies. Separately for quiescent 
and star-forming galaxies, we fit quadratic polynomials to 
the minimum stellar masses versus redshift (galaxies are 
classified into star-forming or quiescent in the following section).
Finally, we use the polynomials to obtain the minimum stellar 
masses at the center of redshift bins, $0.2-0.4$, $0.4-0.6$, 
and $0.6-0.8$, which are then used as PRIMUS stellar mass
completeness limits. 

For our target galaxy population at low redshift we start with 
the SDSS-{\em GALEX} data. We limit our population within 
the redshift range $0.0375-0.145$ due to redshift limits later 
imposed on the volume-limited Environment Defining Population 
(Section \ref{sec:environment}). To account for targeting 
completeness of this sample, we use the statistical weight 
estimates provided by the NYU-VAGC. Then to account for 
stellar mass incompleteness, we impose a uniform stellar mass 
limit of $10^9.8 \mathcal{M}_{\odot}$, a limit determined by the 
stellar mass-to-light ratio completeness of the NYU-VAGC 
sample given the redshift limits. 

The absolute magnitude ($M_{r}$) versus redshift for the 
target galaxy population (black squares) is plotted in Figure 
\ref{fig:targetEDP}. The left-most panel corresponds to the
target sample derived from the SDSS-{\em GALEX} data and
the rest correspond to the target sample derived from the 
PRIMUS data divided in bins with $\Delta z \sim 0.2$. 

\subsection{Classifying Quiescent and Star-Forming Galaxies} \label{sec:sfq}
With the galaxy sample defined in the previous section, we now classify the galaxies as quiescent or star-forming using an evolving cut based on specific star-formation rate utilized in \cite{Moustakas:2013aa} Section 3.2.
To summarize, this classification method utilizes the star-forming (SF) sequence, which is the correlation between star-formation rate (SFR) and stellar mass in star-forming 
galaxies observed at least until $z \sim 2$.
The PRIMUS sample displays a well-defined SF sequence within the redshift range of our target population.
Using the power-law slope for the SF sequence derived by \cite{Salim:2007aa} (SFR $\propto \mathcal{M}^{0.65}$) and the minimum of the quiescent/star-forming bimodality, 
determined empirically, we obtain the following equation to classify the target galaxies (Equation 2 in \cite{Moustakas:2013aa}):
\begin{equation}
\label{eq:qsfclass} 
\rm{log}(\rm{SFR}_{\rm{min}}) = -0.49 + 0.64 \rm{log}(\mathcal{M} - 10) +1.07(z-0.1), 
\end{equation}
where $\mathcal{M}$ is the stellar mass of the galaxy.
If the target galaxy SFR and stellar mass place the galaxy above Equation \ref{eq:qsfclass} we classify it as star-forming; if below, as quiescent (\cite{Moustakas:2013aa} Figure 1.).

\begin{figure*}
  \begin{center}
    \leavevmode
     \epsfig{file=fig_smf_cylr2h25_thresh75_bin0_25_2envbin_primuszerr.eps,height=0.75\textwidth,angle=90}
     \caption{Evolution of stellar mass functions of star-forming (top) and quiescent (bottom) target galaxies in 
low (left) and high (right) environments from redshift range $z=0-0.8$. The environment of each galaxy  
was calculated using a cylindrical aperture size of $R=2 \: \rm{Mpc}$ and $H=25 \: \rm{Mpc}$ and  
classification based on the cut-offs specified in Table \ref{tab:aperture}. The SMFs use mass bins of 
width $\Delta \rm{log}(\mathcal{M}/\mathcal{M}_{\odot})=0.25$. In each panel we use shades of blue 
(star-forming) and orange (quiescent) to represent the SMF at different redshift, higher redshifts being
progressively lighter.}      \label{fig:smf}
    \end{center}
\end{figure*}

\subsection{Galaxy Environment} \label{sec:environment}
In addition to the quiescent/star-forming classification, in this section, 
we measure the environment for our target galaxies. We define the 
environment of a galaxy as the number of neighboring galaxies 
contained within a fixed aperture centered around it. We use a fixed 
aperture environment because, as \cite{Muldrew:2012aa} finds in 
their comparison of different environment definition using simulations, 
it provides a better probe of the entire halo in comparison to other 
environment definitions such as nearest neighbor, which provides 
a better tracer at inter-halo scales. For our aperture, 
we use a cylinder of dimensions: $R_{\rm{ap}} = 2 \; \rm{Mpc}/h$ and 
$H_{\rm{ap}} = 25 \; \rm{Mpc}/h$. Though spherical apertures are 
often used in literature (e.g. \cite{Croton:2005aa}), we use a 
cylindrical aperture in order to account for the PRIMUS redshift errors 
and redshift space distortions (i.e. "Finger of God" effect). \cite{Cooper:2005aa} 
finds that $\pm 1000 \; \rm{km} \rm{s^{-1}}$ optimally reduces the 
effects of redshift space distortions and PRIMUS $\sigma_z \sim 
0.0085$ at $z \sim 0.7$, so aperture height of $25 \; \rm{Mpc}/h$ 
successfully accounts for both of these effects. Furthermore, our 
choice of cylinder radius was motivated by scale dependence 
analyses in literature (\cite{Blanton:2006aa}, \cite{Wilman:2010aa}, 
and \cite{Muldrew:2012aa}), which suggest that galactic properties 
such as color and quenched fractions are dependent on halo-scale 
properties such as host dark matter halo mass. \cite{Wilman:2010aa}, 
which uses environment defined by annuli of different radii, find 
positive correlation for quenched fraction and color on scales 
$< 1 \; \rm{Mpc}$ and anti-correlation on scales $> 3 \; \rm{Mpc}$. 
Our choice of $2 \rm{Mpc}/h$ provides sufficient sample size of 
galaxies in dense environments, for robust statistics, while 
tracing galactic properties within the halo scale. When we extend our 
analysis to cylindrical apertures with $R_{\rm{ap}} = 1 \; \rm{Mpc}/h$, 
we find that the change in $R_{\rm{ap}}$ does not significantly change 
our results (difference in $f_{Q} < 0.05$). 
%Similar fixed aperture methods have also been used in \cite{Croton:2005aa} for galaxies in the 2dF Galaxy Redshift 
%Survey (\cite{Colless:2003aa}) and in \cite{Muldrew:2012aa} for a mock galaxy catalogue generated from embedding 
%galaxies onto the Millenium Dark Matter Simulation (\cite{Springel:2005aa}). 

With proper motivation for the fixed aperture environment definition, 
we now proceed to measuring the environment for our target galaxies. 
First, we construct a volume limited {\em Environment Defining 
Population} (EDP) with absolute magnitude cut-offs ($M_{r}$) at 
each redshift bin. EDP galaxies within the aperture surrounding each 
target galaxy are the neighboring galaxies in our definition of 
environment. The $M_{r}$ cut-offs for the EDP are selected so that 
the cumulative number density over $M_{r}$ at all redshift bins are 
equal. Not only does this provide a reasonably comparison of the 
environment measurements at different redshifts but it also seeks 
to construct an EDP that contains similar galaxy populations 
throughout our redshift range (i.e. to account for the progenitor bias 
in our EDP). As \cite{Behroozi:2013aa} and \cite{Leja:2013aa} find 
in their analysis of the cumulative number density method, the 
cumulative number density method, though it does not precisely 
account for the scatter in mass accretion or galaxy-galaxy mergers, 
provides a reasonable means to compare galaxy populations over 
a wide range of cosmic time. 

In constructing the EDP for the PRIMUS (hereafter PRIMUS EDP) we use the same PRIMUS data as the 
target galaxies, described in Section \ref{sec:target}. 
We restrict the PRIMUS galaxies to $0.2 < z < 0.8$ and divide them into bins of $\Delta z = 0.2$. 
Before we consider the cumulative number densities in these bins, we first determine $M_r$ limit for the 
highest redshift bin ($0.6-0.8$) by examining the $M_{r}$ distribution with bin size $\Delta M_{r} = 0.25$ 
and select $M_{r,\rm{lim}}$ near the peak of the distribution where bins with $M_{r} > M_{r,\rm{lim}}$ 
have fewer galaxies than the bin at $M_{r, \rm{lim}}$. 
We choose $M_{r, \rm{lim}}(0.6 < z < 0.8)$ to be, conservatively, $M_{r} = -20.75$. 
For the other redshift bins, we impose absolute magnitude limits ($M_{r,\rm{lim}}$) such 
that the cumulative number density of the bin ordered by $M_{r}$ is equal to the cumulative 
number density of the highest redshift bin. 
The cumulative number density calculations accounts for the statistical weights of the galaxies (Section 
\ref{sec:primus} and \ref{sec:sdss}).

For the SDSS-{\em GALEX} EDP (hereafter SDSS EDP), we do not use the parent data of 
the SDSS-{\em GALEX} target sample which uses the geometry of the combined angular selection function of the 
SDSS VAGC and {\em GALEX}. 
Instead, since FUV, NUV values are not necessary for EDP, we extend the parent data of the SDSS EDP
to the entire SDSS VAGC, including galaxies outside of the {\em GALEX} window function. 
Furthermore, we impose a redshift range of $0.0375-0.145$ on the SDSS EDP. 
This redshift range is due to the lack of faint galaxies at $z \sim 0.2$ and the lack of bright galaxies at 
$z \sim 0.01$ in the SDSS VAGC data.  
The lower bound for the redshift range was empirically determined by the bright limit and the upper bound 
by the faint limit of the $M_{r}$ versus redshift distribution. 
The same fixed cumulative number density method, described above, is used on this SDSS EDP to equate
the cumulative number density of SDSS EDP to the highest redshift bin of the PRIMUS EDP. 
Ultimately, we get $M_{r,\rm{lim}} = -20.57$, $-20.73$, $-20.80$ and $-20.95$ for the redshift bins $0.0375-0.145$, 
$0.2-0.4$, $0.4-0.6$, $0.6-0.8$, respectively.  %%%% PUT NEW VALUES
These absolute magnitude limits are illustrated in Figure \ref{fig:targetEDP}, which plots $M_{r}$ distribution as 
a function of redshift for the EDP (red) and the target galaxy sample (black). 

Finally with the EDP we measure the environment for each of the target sample galaxies by counting the number
of EDP galaxies, $n_{\rm{env}}$, with $RA$, $Dec$, and redshift within the aperture surrounding it. 
$n_{\rm{env}}$ accounts for the statistical weights of the EDP galaxies. 
%Since we use apertures of different dimensions, we are interested in the relative densitites rather than the actual $n_{env}$ values.
%Hence, we use the percentage rank of the galaxy environment to quantify overdense environments and underdense environments.
%More specifically, for each of the redshift bins ($0.2-0.4$, $0.4-0.6$, $0.6-0.8$, and $0.8-1.0$) the $n_{env}$ values for all target galaxies in that bin are listed and assigned a percentage rank based on their position in the list: $n_{env} = 0$ corresponding to $0\%$ and the maximum $n_{env}$ for a target galaxy in the $z$-bin corresponding to $100\%$. 
%Using these percentage ranks, each target galaxy is classified as high-, mid-, or low-environment based on the cut-offs specified in Table \ref{tab:aperture}. 
%Target sample galaxies are classified as high-environment if its percentage rank lies within the top $20\%$ and as low-environment if its percentage rank lies within the bottom $20\%$. 
%In Table \ref{tab:aperture}, apertures with radius $1 h^{-1} \rm{Mpc}$ have a low-eneivonrment cut-off of higher than $20\%$.
%This is because over $20\%$ of target sample galaxies have $n_{env}=0$ such a smaller aperture size.
%Hence we defined the low-environment percentage rank cut-off to contain all galaxies with $n_{env}=0$. 
%In order to have a fair comparison for the different $z$-bins when using this aperture, the low-environment cut-off was selected as the lowest cut-off that includes all galaxies with $n_{env}=0$ for all $z$-bin.
More specific details for the dense and sparse environment cut-offs the various apertures are provided in Table \ref{tab:aperture}.

\begin{figure*}
    \begin{center}
        \leavevmode
        \epsfig{file=fig_qf_cylr2h25_thresh75_bin0_25_primuszerr.eps,height=0.75\textwidth,angle=90}
        \caption{Evolution of the quiescent fraction $f_{\rm{Q}}$ for target galaxies in spare (left) and dense (rights) environments
from $z \sim 0.7$ to $z \sim 0.1$. $f_{\rm{Q}}$s were calculated using the SMFs computed in Section \ref{sec:smf} and shown in Figure\ref{fig:smf}, as described in text. Darker shading indicates lower redshift.}         \label{fig:qf}
    \end{center}
\end{figure*}

\subsection{Edge Effects} \label{sec:edgeeffect}
One of the challenges in obtaining the galaxy environment using a fixed aperture method is accounting for the edges of the survey.
For galaxies located near the edge of the survey, part of the fixed aperture encompassing it will lie outside the survey regions. 
In this case, the $n_{env}$ will only reflect the fraction of the environment within the survey geometry.

In order to account for these edge effects, we use a Monte Carlo method to impose edge cuts on the target 
galaxy population. 
We begin by computing the angular separation, $\theta_{\rm{ap}}$ that corresponds to the radius of the 
aperture at the redshifts of the target galaxies.
Then the galaxies are matched to a sample of $N_{\rm{ransack}}=1,000,000$ points with $RA$ and $Dec$
 randomly generated within the window function of the EDP.  
We refer to this randomly generated redshift-less sample as the "ransack" sample, based on the procedure 
used to construct them. 
For each target galaxy, we count the number of ransack points, $n_{\rm{ransack}}$, within $\theta_{\rm{ap}}$
of the galaxy's $RA$ and $Dec$ value.
The $n_{\rm{ransack}}$ values are then compared to the expected value:
\begin{equation} \label{eq:ransack}
E[n_{\rm{ransack}}] = \frac{N_{\rm{ransack}}}{A_{\rm{EDP}}}\times {\pi \theta_{\rm{ap}}^2} \times f_{\rm{thresh}} 
\end{equation} 
where $A_{\rm{EDP}}$ is the total angular area of the target fields and $f_{\rm{thresh}}$ is the fractional 
threshold for the edge effect cut-off, which we vary based on $R_{\rm{ap}}$ (listed in Table \ref{tab:aperture}).
If $n_{\rm{ransack}}$ for a target galaxy is greater than $E[n_{\rm{ransack}}]$ then the target galaxy remains
in the sample; otherwise, it is discarded from the sample. 

%\section{Results}
%In the section below, we calculate stellar mass functions (SMFs) for the target galaxy population divided into 
%quiescent/star-forming and dense/sparse environment, Section \ref{sec:smf_const}. We use a non-parametric 
%$1/V_{\rm{max}}$ estimate of SMFs for the subsamples further divided into redshift bins. Then using these 
%SMFs, we calculate the quiescent fractions (QFs), Section \ref{sec:qf_const}, 
%and examine it's evolution over our redshift range. 

\section{Stellar Mass Function} \label{sec:smf}
The target galaxy sample defined above has so far
been classified into quiescent or star-forming and 
dense or sparse. We further divide these subsamples 
into bins of redshift: $0.0375-0.145$, 
$0.2-0.4$, $0.4-0.6$, and $0.6-0.8$. While the 
PRIMUS data ranges from $0.2 < z < 1.2$, we only 
consider galaxies with $ z<0.8$ due to insufficient
statistics for robust environment measurements at 
$z > 0.8$. With added redshift bins, we have a
total of 16 subsamples. We calculate the SMF for 
each of these 16 subsamples. 
%% EMPHASIZE THE FACT THAT WE HAVE ENVIRONMENT MEASURES AND A ROBUST ENOUGH DATA TO PROBE 
%% UP TO REDSHIFT 0.7! THAT WAY WE CAN SEE THE MASS DISTRIBUTION OF VERY SPECIFIC POPULATIONS OF
%% GALAXIES UP TO A PRETTY HIGH REDSHIFT!

To calculate the SMFs we employ a non-parametric 
$1/{V_{\rm{max}}}$ estimator commonly used for 
galaxy luminosity functions and stellar mass functions
, as done in \cite{Moustakas:2013aa} and discussed 
in the review \cite{Johnston:2011aa}. The differential 
SMF is given by the following equation:
\begin{equation} \label{eq:phi}
\Phi(\rm{log}\: \mathcal{M}) \Delta(\rm{log} \:\mathcal{M}) = \sum\limits_{i=1}^{N} \frac{w_i}{V_{\rm{max,avail},i}}. 
\end{equation}
The equation above is same as Equation 3. in 
\cite{Moustakas:2013aa} except for the distinction that 
we use $V_{\rm{max,avail}}$ instead than $V_{\rm{max}}$, 
to account for the edge effects of the survey for our
environment measurements. $w_i$ here represents 
the statistical weight of each galaxy $i$ and 
$\Phi(\rm{log}\: \mathcal{M}) \Delta(\rm{log}\: \mathcal{M})$ 
is the number of galaxies ($N$) per unit volume within the 
stellar mass range 
$[\rm{log} \:\mathcal{M}, \rm{log} \:\mathcal{M}+\Delta(\rm{log} \:\mathcal{M})]$.

$V_{\rm{max},i}$ is the maximum cosmological volume 
where it is possible to observe galaxy $i$ given the apparent 
magnitude limits of the survey. However in Section \ref{sec:edgeeffect} 
we remove the galaxies that lie on the edge from our sample. 
In doing so we reduce the maximum cosmological volume 
where a galaxy can be observed, thereby reducing $V_{\rm{max},i}$.

To calculate $V_{\rm{max,avail},i}$, we generate a sample of 
points with random $RA$, $Dec$ within the window function of 
the target sample and $z$ in the redshift range. This is not to be 
confused with the ransack sample in Section \ref{sec:edgeeffect}. 
We then impose the same edge cuts we applied to the target 
galaxy population. At redshift bins of $\Delta z \sim 0.01$, we 
compute theraction of random points that remain in the bin after 
the edge cuts: $f_{\rm{edge}}$. Afterwards we apply this factor to 
compute $V_{\rm{max,avail}} = V_{\rm{max}} \times f_{\rm{edge}}$. 
The $V_{\rm{max}}$ in the equation above are computed following 
the method described in \cite{Moustakas:2013aa} Section 4.2 with 
the same redshift-dependent $K$-correction from observed SED 
and luminosity evolution model.

In order to calculate the uncertainty of the SMFs from the sample 
variance, we use a standard jackknife technique as done in \cite{Moustakas:2013aa}.
For the PRIMUS target galaxies, we calculate SMFs after excluding 
one of the five target fields each time. And for the SDSS target galaxies
we divide the field into a 30 $\times$ 20 rectangular $RA$ and $Dec$ 
grid and calculate the SMFs after excluding part of the field each time. 
Then using the calculated SMFs we calculate the uncertainty: 
\begin{equation}
\sigma^j = \sqrt{\frac{M-1}{M} \sum\limits_{k=1}^{M} (\Phi^j_k - \langle \Phi^j \rangle)^2}
\end{equation} 
$M$ in this equation is the number of jack knife SMFs in the stellar mass
bins. $\langle \Phi^j \rangle$ is the mean number density of galaxies 
in each stellar mass bin for all of the jack knife $\Phi^j$s. 

SMFs for 16 target galaxy subsamples classified into quiescent/star-forming (orange/blue)and 
dense/spare environments are presented in Figure \ref{fig:smf}. The 
redshift evolution of the SMFs are indicated by a darker shade for lower
redshifts. The sample variance uncertainties are represented by the 
width of the SMFs. The environment measurements and classifications
for Figure \ref{fig:smf} are done using a cylindrical aperture with dimensions,
$R_{\rm{ap}} = 2 \; \rm{Mpc}/h$ and $H_{\rm{ap}} = 25 \; \rm{Mpc}/h$. 

Examining the SMF evolution from $z \sim 0.7$ to $z \sim 0.1$, 
we find that in dense environments both the star-forming and quiescent
SMFs increase at $\mathcal{M}_{*}$ below the knee of mass function 
($\rm{log}\: \mathcal{M}_{*}/\mathcal{M}_{\odot} < 10.75$ for star-forming; 
$\rm{log}\: \mathcal{M}_{*}/\mathcal{M}_{\odot} < 11.0$ for quiescent). 
Meanwhile, at masses above the knee the SMFs for dense environments 
exhibit little change throughout the redshift range.

At sparse environments, the quiescent SMF remains relatively constant 
for galaxies with $\rm{log}\: \mathcal{M}_{*}/\mathcal{M}_{\odot} < 10.75$.
At higher stellar masses, however, the SMF decreases notably. Finally 
the sparse environment star-forming SMF shows a decrease at all 
stellar mass ranges over the redshift range. 

%%% COMPARISON TO OTHER LITERATURE? 

\section{Quiescent Fraction} \label{sec:qf_const}
The SMFs calculated in the previous section describe
the distribution of our galaxy population in stellar mass 
and reveal the evolution of this distribution
over comic time. From the SMFs, we 
compute the quiescent fractions in order to compare the 
quiescent and the star-forming populations 
and to quantify the fraction of galaxies that have
depleted their star-formation. Furthermore, by dividing our 
galaxy sample using environment measurements into dense
and sparse samples, we investigate the environment 
dependence on the quiescent fraction and consequently 
on environment dependent quenching mechanisms. 

From the SMF number densities ($\Phi$)
in the previous section, the quiescent fraction is computed 
as follows, 
\begin{equation}
f_{\rm{Q}} = \frac{\Phi_{Q}}{\Phi_{SF}+\Phi_{Q}}.
\end{equation}
$\Phi_{Q}$ and $\Phi_{SF}$ are the total number of galaxies 
per unit volume in stellar mass bin of $\Delta(\rm{log} \: \mathcal{M}) = 0.25$ 
for the quiescent and star-forming subsamples, respectively 
(Equation \ref{eq:phi}). We compute $f_{\rm{Q}}$ for dense 
and spare environments over our redshift range as plotted in
 Figure \ref{fig:qf}, which shows the evolution of $f_{\rm{Q}}$ 
 for dense (right) and sparse (left) environments. As in 
 Figure \ref{fig:smf}, the darker shading represent lower redshifts. 

In both sparse and dense environments, Figure \ref{fig:qf}
clearly shows an increase in $f_{\rm{Q}}$ with decrease in redshift at all 
stellar masses. Qualitatively, this $f_{\rm{Q}}$ evolution 
exhibits a notable mass dependence. More specifically, 
the $f_{\rm{Q}}$ in sparse environments shows a greater 
evolution between redshift bins at high masses than at low 
masses. On the other hand, the $f_{\rm{Q}}$ in dense 
environments shows a greater evolution between redshift 
bins at low masses than at high masses.

\begin{figure}
    \begin{center}
        \leavevmode
        \epsscale{1.0}
        \epsfig{file=fig_qffit_cylr2h25_thresh75_bin0_25_fidmass10_5_primuszerr.eps,height=0.45\textwidth}
        \caption{The evolution of the quiescent fraction at fiducial mass, $f_{Q}(\mathcal{M}_{\rm{fid}} = 10^{10.5} \mathcal{M}_\odot)$, for sparse (square) and dense (circle) environments within the redshift range $z = 0.0 - 0.8$.  There is a significant increase in $f_{Q}(\mathcal{M}_{\rm{fid}})$ with decrease in redshift for both environments. In addition, over the entire redshift range, $f_{Q}(\mathcal{M}_{\rm{fid}})$ for dense environment is greater than $f_{Q}(\mathcal{M}_{\rm{fid}})$ for lower environment. However the difference in $f_{Q}(\mathcal{M}_{\rm{fid}})$ for the two environments remains constant throughout suggesting that while the quiescent fraction is higher in dense environments, the evolution of the quiescent fraction is independent of environment.}         \label{fig:qffit}
    \end{center}
\end{figure}
%fiducial mass may be 11.0 depending on what we want.
% fixed slope? Can't do fixed slope if we say there is mass dependence in f_q evolution

While trends are apparent from Figure \ref{fig:qf}, quantitative comparisons of the 
$f_{\rm{Q}}$ for different environment over redshift is made challenging by the distinct stellar mass 
completeness limits for each redshift bin. In order to quantify the quiescent fraction over the stellar mass complete rate, we fit each $f_{\rm{Q}}$ to a power-law parameterization as a function of stellar mass, 
\begin{equation} 
f_{\rm{Q}}(\mathcal{M}_{*}) = a \: \rm{log} \; \left(\frac{ \mathcal{M}_{*}}{10^{10.5} \: \mathcal{M}_{\odot}} \right)+b,
\end{equation}
where $a$ are $b$ are best-fit parameters using {\em MPFIT} (\cite{Markwardt:2009aa}).
The value $10^{10.5} \: \mathcal{M}_{\odot}$ in the equation represents an empirically selected
fiducial mass $\mathcal{M}_{\rm{fid}}$ within the stellar mass completeness limits. This fiducial mass 
serves to highlight and quantify the $f_{\rm{Q}}$ evolution for different redshifts and using $\mathcal{M}_{\rm{fid}} = 10^{11} \mathcal{M}_\odot$ 
does not notably alter the results. Figure \ref{fig:qffit} shows the evolution of $f_{\rm{Q}}(\mathcal{M}_{\rm{fid}})$ from $z \sim 0.7$ to $\sim 0.1$ 
for sparse (diamond) and dense (circle) environments.
For both dense and spare environments, $f_{\rm{Q}}(\mathcal{M}_{\rm{fid}})$ increases as redshift decreases. In addition, through the redshift range explored in our analysis, dense environment $f_{\rm{Q}}(\mathcal{M}_{\rm{fid}})$ is significantly greater than the sparse environment $f_{\rm{Q}}(\mathcal{M}_{\rm{fid}})$. However when we compute $f_{\rm{Q}}(\mathcal{M}_{\rm{fid}})_{\rm{dense}} - f_{\rm{Q}}(\mathcal{M}_{\rm{fid}})_{\rm{sparse}}$, we find the difference remains constant throughout the redshift range ($ < 0.15$ throughout redshift range). Furthermore, the total evolution of $f_{\rm{Q}}(\mathcal{M}_{\rm{fid}})$ from $z \sim 0.7$ to $\sim 0.1$ show no strong environment dependence. 

The analysis described in this paper use a fixed cylindrical aperture with dimensions $R_{\rm{ap}} = 2 \: \rm{Mpc}$ and $H_{\rm{ap}} = 25 \: \rm{Mpc}$ to measure environment, the same analysis was extended for varying aperture dimensions $R_{\rm{ap}} = 1, 2, 3 \: \rm{Mpc}$ and $H_{\rm{ap}} = 25, 50 \: \rm{Mpc}$. 
Minor adjustments to the environment classification thresholds were adopted in these analyses for the smaller apertures ($r_{\rm{ap}} = 0.5, 1 \rm{Mpc}$ and $r_{\rm{ap}} = 25 \rm{Mpc}$).
The results obtained from using these different are consistent with the results displayed in this paper. 

\section{Summary}
We have measured the SMFs and QFs using low redshift SDSS-{\em GALEX} galaxies and intermeidate redshift PRIMUS galaxies. 
Specifically we anlayzed the evolution of the QFs over the redshift range $0.0-1.0$ for galaxies in environment densities (Figure \ref{fig:qf}). 
We find that there is an expected increase in QF with decrease in the redshift for subsamples in all environment densities.
More importantly we find that the change in QF over redshift is independent of the environment and remains relatively equal for all environments. 

\begin{itemize}
    \item Comparison to other works. 
    \begin{itemize}
        \item Alberts et al. 2013 
    \end{itemize}
\end{itemize}

%
% References
%
\bibliography{PRIMUS}

\appendix
\begin{table*} %Will contain much more information on environments. 
  \caption{Fixed Cylindrical Aperture Dimensions}
  \label{tab:aperture}
  \begin{center}
    \leavevmode
    \begin{tabular}{llllll} \hline \hline              
  Radius (Mpc)          &Height (Mpc)      &$n_{bin}$   &Edgecut &High Env Threshold (galaxies) &Low Env Threshold (galaxies) \\ \hline 
  1.0 &50 &6 & 80\% & 1.5 & 0.0          \\
  2.0 &50 &6 & 75\% & 4.0 & 0.0          \\ \hline
  \multicolumn{6}{l}{}                                             \\       
    \end{tabular}
  \end{center}
\end{table*}

\subsection{Stellar Mass Function} \label{sec:smf_const}
\begin{figure*}
    \begin{center}
        \leavevmode
        \epsscale{1.0}
        \plotone{fig_smf_cylr1h25_thresh80_bin0_25.png}
        \caption{SMF for $r_{\rm{ap}}=1 \rm{Mpc}$ and $h_{\rm{ap}}=25 \rm{Mpc}$}
    \end{center}
\end{figure*}

\begin{figure*}
    \begin{center}
        \leavevmode
        \epsscale{1.0}
        \plotone{fig_qf_cylr1h25_thresh80_bin0_25.png}
        \caption{QF for $r_{\rm{ap}}=1 \rm{Mpc}$ and $h_{\rm{ap}}=25 \rm{Mpc}$}
    \end{center}
\end{figure*}

\begin{figure*}
    \begin{center}
        \leavevmode
        \epsscale{1.0}
        \plotone{fig_qffit_cylr1h25_thresh80_bin0_25.png}
        \caption{QF at fiducial mass for $r_{\rm{ap}}=2 \rm{Mpc}$ and $h_{\rm{ap}}=50 \rm{Mpc}$}
    \end{center}
\end{figure*}

\end{document}
